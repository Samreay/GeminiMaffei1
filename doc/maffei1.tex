\documentclass[useAMS,usenatbib]{mn2e}
\usepackage{times,graphicx,hyperref}
\usepackage{amssymb}

%%%%% AUTHORS - PLACE YOUR OWN MACROS HERE %%%%%
\def\farcs{\hbox{$.\!\!^{\prime\prime}$}}
\def\farcsh{\hbox{$.\!^{\mathrm{s}}$}}
\def\mnras{MNRAS}
\def\nat{Nature}
\def\aap{A\&A}
\def\apj{ApJ}
\def\apjl{ApJ}
\def\aj{AJ}
\def\pasj{PASJ}
\def\pasp{PASP}
\def\apjs{ApJS}
\def\aaps{A\&AS}
\def\aplett{ApL}
\def\pasa{PASA}

%%%%%%%%%%%%%%%%%%%%%%%%%%%%%%%%%%%%%%%%%%%%%%%% 

\title[Globular clusters in Maffei 1]{A wide-field study of globular
  clusters in the nearest giant elliptical: Subaru/Suprime-Cam
  observations of Maffei 1\thanks{Based on data collected at Subaru
    Telescope, which is operated by the National Astronomical
    Observatory of Japan.}}  

\author[S. Hinton et al.]{
Samuel Hinton$^{1,2}$, 
Ricardo Salinas$^{3}$% \thanks{E-mail: rsalinas@gemini.edu}, 
Aaron J. Romanowsky$^{4,5}$, and maybe some others
\\
$^1$School of Mathematics and Physics, University of Queensland, QLD 4072, Australia \\
$^2$Australian Astronomical Observatory, North Ryde, NSW 2113, Australia \\
$^3$Gemini Observatory\\ 
$^4$Department of Physics \& Astronomy, San Jos\'e
  State University, San Jose, CA 95192, USA\\ 
$^5$University of California Observatories,
1156 High Street, Santa Cruz,CA 95064, USA
}
\begin{document}

\date{Accepted ... Received ...; in original form ...}


\pagerange{\pageref{firstpage}--\pageref{lastpage}} \pubyear{2016}

\maketitle

\label{firstpage}
%%%%%%%%%%%%%%%%%%%%%%%%%%%%%%%%%%%%%%%%%%%%%%%

\begin{abstract}
Lorem ipsum dolor sit amet, consectetur adipiscing elit. Sed condimentum ipsum faucibus sem elementum, eu consectetur risus consectetur. Praesent quis aliquam risus. Vivamus interdum et eros gravida rhoncus. Proin accumsan finibus pellentesque. Praesent et erat eu velit maximus condimentum. Integer ornare vestibulum dolor nec pretium. Suspendisse suscipit, metus sed consectetur pretium, ligula nunc mollis quam, id dapibus est tortor et massa. Maecenas maximus elit orci, sit amet tincidunt libero ultrices a. Nullam sit amet condimentum libero. Duis ullamcorper lorem in nisi gravida tincidunt. Aliquam erat volutpat. Sed vitae sem a elit mattis porttitor.
\end{abstract}

\begin{keywords}
Galaxies: individual: Maffei 1 -- Galaxies: star clusters
\end{keywords}

%%%%%%%%%%%%%%%%%%%%%%%%%%%%%%%%%%%%%%%%%%%%%%%%
\section{Introduction}
\label{sec:intro}

{\bf Add some general remarks on globular cluster systems.}
 
Maffei 1 is a giant elliptical galaxy \citep{maffei68,spinrad71}
located at only 0.5\degr\,North of the Galactic plane. This position
implies large obscuration by dust from the Galactic disc which has
hampered the measurement of even the most basic parameters of the
galaxy.

{\bf Give main results from the literature}

Its particular relative position has also conspired against the study
of its globular clusters (GCs). Studies of its globular cluster system
(GCS) have produced only a handful of GC candidates
\citep{davidge02,buta03,davidge05}, whereas for its luminosity
\citep[$M_V\sim-20.80$,][]{fingerhut07}, the size of its GCS should be
comparable to the one of Cen A, hosting $\sim 1300$ GCs
\citep{harris10}. 

In this paper we present the first wide-field study of the Maffei 1
GCS system using Subaru/SuprimeCam imaging.In Sect. \ref{sec:obs} we
present the imaging data used, as well as its reduction and
photometry. In Sect. \ref{sec:results} we present the GCs selection,
together with the derived properties of the
GCS. Sect. \ref{sec:discussion} puts our results on a wider context,
while Sect. \ref{sec:conclusions} summarizes the work and give
conclusions.

%%%%%%%%%%%%%%%%%%%%%%%%%%%%%%%%%%%%%%%%%%%%%%%%%%%%%%%%%%%%%%%%%%%%
\section{Subaru/Suprime-Cam observations and data reduction}
\label{sec:obs}

Maffei 1 images were obtained using Suprime-Cam \citep{miyazaki02}
located on the Subaru telescope, Mauna Kea, Hawaii. Suprime-Cam
comprises 10 CCD detectors separated by $\sim15\arcsec$ covering a
field-of-view of $34\arcmin\times27\arcmin$ with a pixel scale of
0.2\arcsec. SDSS $r'$, $i'$ and $z'$-band images were obtained during
the night of January 5th, 2011.  Several short exposures were taken
with a $\sim\!  1\arcmin$ dither pattern, totalling 385, 280, and 280
seconds in the $r'$, $i'$ and $z'$-bands, respectively.

Suprime-Cam images reduction was conducted within the \textsc{sdfred2}
pipeline \citep{ouchi04}. Reduction steps include bias subtraction,
flat-fielding, correction for atmospheric distortions, point spread
function (psf) equalization (i.e. the normalization of the psf to a
single shape across the detectors and exposures), sky subtraction,
image alignment and finally, the combination of all exposures and
detectors into single images. Final averaged images have a
field-of-view of $\sim37\arcmin\times 31\arcmin$ and a seeing quality
of 0.72\arcsec, 0.68\arcsec and 0.63\arcsec, for $r'$, $i'$ and $z'$,
respectively.

For the $z'$ image set, which is used as the basis for the GC
candidates identification given its higher image quality (see
Sect. \ref{sec:gc_candidates}), a parallel alternative reduction
process was adopted. The $z'$ dataset not only provides the reddest
band to pierce through the Galactic plane, but also contains the best
seeing images. Even though the same \textsc{sdfred2} pipeline was
mostly used, a couple of the aforementioned reduction steps were
skipped in order to mantain image manipulations that could alter the
image quality to a minimum. Firstly, given the slight degradation of
the psf towards the outer detectors, psf equalization across the chips
would imply a lost of information on the best chips. For this reason
the detectors were reduced independently and not combined into a
single image as a final step. As a second difference, we did not apply
any correction for atmospheric distorsions, since it resulted in an
increase of about 10\% of the measured stellar full width at half
maximum (fwhm) due to charge shifting to neighbor pixels. Finally,
images with psf sizes significantly larger than the mean were excluded
from the final image combination. This resulted in the rejection of
zero to two exposures per chip.

Coordinate transformations and image combination for the individual
detectors were carried out with \textsc{daomaster/montage2}
\citep{stetson93,stetson94}, which give more flexibility than
\textsc{sdfred2} for images with a small overlapping area. Measured
fwhm on the combined $z'$ images varies between 0.52\arcsec to
0.58\arcsec from detector to detector, which are noticeably better
than the 0.62\arcsec measured on the combined full frame image given
by \textsc{sdfred2}.

As a last step prior to photometry, images presenting large background
variations (containing Maffei 1 light or Galactic cirrus), were
medianed filtered with 128 pixel boxes; a large size that preserves
point-like sources unaltered. Photometry and astrometry were performed
on the images fully processed with the \textsc{sdfred2} pipeline.

\begin{figure}
%\includegraphics[width=0.49\textwidth]{maffei1_fig1.ps}
\caption{The Maffei 1 field as seen by Subaru-SuprimeCam. Image size
  is approximately $37\arcmin\times 31\arcmin$. North is up and East is
  to the left.}
\label{fig:maffei1}
\end{figure}


\subsection{Stellar(ish) photometry}
\label{sec:photometry}

Aperture and psf-fitting photometry were carried out using the
stand-alone \textsc{daophot2} photometric suite \citep{stetson87}. The
psf was modelled selecting $\sim$120--140\, bright and isolated stars
on each detector. A quadratically varying Moffat function was found to
provide the best description of the psf behaviour within the
detectors.

Photometry for all the sources was also obtained using
\textsc{SExtractor} \citep{bertin96}, which additionally provides
measurements of the size, orientation and ellipticity of the sources.

%%%%%%%%%%%%%%%%%%%%%%%%%%%%%%%%%%%%%%%%%%%%%%%%%%%%%%%%%%%%%%%%%%%%%%
\section{Results}
\label{sec:results}
\subsection{Selecting globular clusters candidates}
\label{sec:gc_candidates}

The basic selection criterium for GCs in Maffei 1 rests on the
non-stellar shape of their light profiles. This shape, usually smeared
out for distant systems or in images with low spatial resolution,
should be noticeable for the largest GCs in Maffei 1 given the
galaxy's distance and the quality of the imaging. A similar approach
has been applied succesfully on ground-based imaging of GCs in the
slightly more distant elliptical galaxy Centaurus A
\citep{rejkuba01,gomez06,gomez07}.

Star-subtracted $z'$-images produced by the psf-fitting algorithm in
\textsc{allstar}, were visually inspected to look for residuals that
revealed the presence of extended sources. The light profile of GCs
(but also galaxies) would be over-subtracted in the central parts and
under-subtracted in the wings, leaving an easily recognizable
``ring-shaped'' residual (see Fig. \ref{fig:subtraction}). A visual
inspection of the residuals in the entire SuprimeCam field of view is
painstaking proccess, but it was preferred over more automatized
procedures such as relying on the star-galaxy separation of
\textsc{SExtractor} that could result in a large amount of false
detections; a posterior analysis has shown that clearly resolved
sources have been assigned a \verb+CLASS_STAR+ parameter value as high
as 0.98, which on a blind approach would have been classified as
stars. Our selected approach revealed the presence of XXX extended
sources in the entire field of view.

The extended sources were separated into 3 classes based on the visual
appeareance of their residuals in combination with their structural
parameters measured with \textsc{SExtractor}: circular residuals with
fwhm $< XX$ (class A), symmetric but elongated residuals where the
source has $\epsilon< 0.3$ (class B), and finally, elongated residuals
with $\epsilon > 0.3$ together with sources with very extended (fwhm$
> 6$ pixels) or asymmetric residuals (class C). In the few cases where
the source was not detected by SExtractor (usually because of the
proximity of a bright star or the patchy Maffei 1 center), the
classification was made only based on their visual appearance. The
ellipticity limit of 0.3 was chosen since it contains all the
ellipticities found for GCs in Local Group galaxies
\citep[e.g.][]{vdb08}. These classes correlate with the likelihood of
the sources being genuine Maffei 1 GCs based on their structure, with
the round class-A objects having a higher probability of being genuine
Maffei 1 GCs, and the elongated or large class-C sources, being most
probably background galaxies.

\begin{figure}
%\includegraphics[width=0.49\textwidth]{maffei1_fig2.ps}
\caption{Two examples of the psf-subtraction technique applied on the
  final $z$ images to reveal extended sources. The left panels show
  the original image, while the right panels are the psf-subtracted
  images. Top panels: A round ``Class A'' object with fwhm=3.4
  pixels. Note the ring-shaped residual. Bottom panels: A somewhat
  elongated ``Class B'' object with $\epsilon=0.12$.}
\label{fig:subtraction}
\end{figure}

\subsection{Globular clusters color-magnitude diagram}
\label{sec:gc_cmd}

\subsection{Globular cluster sizes}
\label{sec:gc_sizes}


%%%%%%%%%%%%%%%%%%%%%%%%%%%%%%%%%%%%%%%%%%%%%%%%%%%%%%%%%%%%%%%%%%%%%%
\section{Discussion}
\label{sec:discussion}

%%%%%%%%%%%%%%%%%%%%%%%%%%%%%%%%%%%%%%%%%%%%%%%%%%%%%%%%%%%%%%%%%%%%%%
\section{Summary and conclusions}
\label{sec:conclusions}

%%%%%%%%%%%%%%%%%%%%%%%%%%%%%%%%%%%%%%%%%%%%%%%%%%%%%%%%%%%%%%%%%%%%%%%%

\section*{Acknowledgments}

AJR was supported by National Science Foundation grants AST-0808099
and AST-0909237.

\bibliographystyle{mn2e}
\bibliography{maffei1}

\appendix
\section{Globular cluster candidates photometry}
\label{sec:appendix}
\begin{table*}
 \centering
 \caption{Photometry of ``Class A'' (see main text for its definition) globular cluster candidates.}
\label{tab:class_a}  
\begin{tabular}{@{}lllccccl@{}}
  \hline
ID & RA & Dec &$r'$& $i'$& $z'$& $\epsilon$& Notes\\
GC1 & XX:XX:XX.XX & XX:XX:XX.XX & $99.99 \pm 99.99$ & $99.99 \pm 99.99$ & $99.99 \pm 99.99$ & $ 9.99$& \\
 \hline

\hline
\end{tabular}
\end{table*}

\begin{table*}
 \centering
  \caption{Photometry of ``Class B'' globular cluster candidates.}
\label{tab:class_b}  
\begin{tabular}{@{}lllccccl@{}}
  \hline
ID & RA & Dec &$r'$& $i'$& $z'$& $\epsilon$& Notes\\
 \hline

\hline
\end{tabular}
\end{table*}

\begin{table*}
 \centering
  \caption{Photometry of ``Class C'' globular cluster candidates.}
\label{tab:class_c}  
\begin{tabular}{@{}lllccccl@{}}
  \hline
ID & RA & Dec &$r'$& $i'$& $z'$& $\epsilon$& Notes\\
 \hline

\hline
\end{tabular}
\end{table*}

\bsp
\label{lastpage}

\end{document}
